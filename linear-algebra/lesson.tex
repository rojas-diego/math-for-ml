\documentclass[12pt]{article}

%Packages%
\usepackage{amsmath, bm, xcolor, amssymb, parskip, cancel}

%Meta%
\title{Linear Algebra}
\author{Diego ROJAS}

%Shortcuts%
\newcommand{\R}{{\rm I\!R}}
\newcommand{\N}{{\rm I\!N}}
\newcommand{\bx}{{\bm{x}}}
\newcommand{\A}{{\bm{A}}}
\newcommand{\B}{{\bm{B}}}
\newcommand{\vecs}[2]{{\bm{#1}_1, \dots, \bm{#1}_#2}}
\newcommand{\xdefinition}[2]{\paragraph{\colorbox{#1!30}{\textbf{Definition:}} (#2):}}
\newcommand{\xrule}[1]{\paragraph{\colorbox{#1!30}{\textbf{Rule:}}}}
\newcommand{\mc}[1]{\mathcal{#1}}

%Document%
\begin{document}

\maketitle

%Introduction%	
\section{Introduction}

Linear algebra is a discipline of mathematics that deals with linear equations, vectors, matrices and their transformations.

%Systems of linear equations%
\section{Systems of linear equations}

%Matrices%
\section{Matrices}

%Solving systems of linear equations%
\section{Solving systems of linear equations}

%Vector spaces%
\section{Vector spaces}

\xdefinition{blue}{Group} Given a set $\mc{G}$ and an operation $\otimes$ defined on $\mc{G}$. $G := (\mc{G}, \otimes)$ is a group if the following holds:
%
\begin{itemize}
	\item Closure of $\mc{G}$ under $\otimes$: $\forall x, y \in \mc{G} : x \otimes y \in \mc{G}$
	\item Associativity: $\forall x, y, z \in \mc{G} : (x \otimes y) \otimes z = x \otimes (	y \otimes z)$
	\item Neutral element: $\exists e \in \mc{G} \ \forall x \in \mc{G} : x \otimes e = x$
	\item Inverse element: $\forall x \in \mc{G} \ \exists y \in \mc{G}: x \otimes y = e$ where $e$ is the neutral element.
\end{itemize}
%

\xdefinition{red}{Vector space} A real-valued vector space $V = (\mc{V}, +, \cdot)$ is a set $\mc{V}$ with two operations $+$ and $\cdot$ defined where the following holds:
%
\begin{itemize}
	\item $(\mc{V}, +)$ is an abelian group.
	\item Distributivity:
	\begin{itemize}
	\item $\forall \bm{x}, \bm{y} \in \mc{V} \ \forall \lambda \in \R : \lambda(\bm{x} + \bm{y}) = \lambda \bm{x} + \lambda \bm{y}$
	\item $\forall \bm{x} \in \mc{V} \ \forall \lambda, \psi \in \R : (\lambda + \psi) \bm{x} = \lambda \bm{x} + \psi \bm{y}$
	\end{itemize}
	\item Associativity (outer operation)
	\item Neutral element (outer operation)
\end{itemize}
%

\xdefinition{red}{Vector subspace} Let $V = (\mc{V}, +, \cdot)$ be a vector space and $\mc{U} \subseteq \mc{V}, \mc{U} \neq \cancel{0}$. Then $U = (\mc{U}, +, \cdot)$ is a \textit{vector subspace} (or \textit{linear subspace}) if $U$ is a vector space restricted to $\mc{U} \times \mc{U}$ and $\mc{U} \times \R$. We write $U \subseteq V$.

\xrule{red} Given $U \subseteq V$, the following properties of $V$ are passed to $U$.
%
\begin{itemize}
	\item Abelian group, distributivity, associativity and neutral element properties

	\item $\mc{U} \neq \cancel{0}$, in particular: $\bm{0} \in \mc{U}$
	\item Closure of $U$
	\begin{itemize}
		\item $\forall \lambda \in \R \ \forall \bm{x} \in \mc{U} : \lambda \bm{x} \in \mc{U}$
		\item $\forall \bm{x}, \bm{y} \in \mc{U} : \bm{x} + \bm{y} \in \mc{U}$
	\end{itemize}
\end{itemize}

\xrule{red} Every subspace $U \subseteq (\R^n, +, \cdot)$ is the solution space of a homogeneous system of linear equations $\A \bm{x} = \bm{0}$ for $\bm{x} \in \R^n$

%Linear independence%
\section{Linear independence}

\xdefinition{red}{Linear combination} Consider a vector space $V$ and a finite number of vectors $\bx_1, \dots, \bx_k \in V$. Then every $\bm{v} \in V$ of the form:
%
	\begin{equation}
		\bm{v} = \sum\limits_{i=1}^{k} \lambda_i \bm{x}_i \in V
	\end{equation}
%
	with $\lambda_1, ÷\dots, \lambda_k \in \R$ is a linear combination of the vectors $\bx_1, \dots, \bx_k$.

\xdefinition{red}{Linear independence} Given a vector space $V$ and $k \in \N$. Vectors $\bx_1, \dots, \bx_k \in V$ are said to be linearly independent if there exists no non-trivial solution to $\bm{0} = \sum\nolimits_{i=1}^{k} \lambda_i \bx_i$ with at least one $\lambda_i \neq 0$. Otherwise they're linearly dependent.

\xrule{red}
%
\begin{itemize}
  \item If at least one of the vector $\bm{x}_i$ is $\bm{0}$ then they are linearly dependent.
  \item The vectors $\{\bx_1, \dots, \bx_k : \bx_i \neq \bm0, i =1, \dots, k\} , k \geqslant 2$ are linearly dependent, if and only if, at least one of them is a linear combination of the others.
  \item To check whether $\bx_1, \dots, \bx_k \in V$ are linearly independent we can use Gaussian elimation: write all vectors as columns of a matrix $\A$ and perform Gaussian elimination until the matrix is in row-echelon form.\begin{itemize}
  \item Non-pivot columns can be expressed as linear combinations of vectors on their left.
  \item Pivot columns are linearly independent from vectors on their left.
	  \end{itemize}
	  If all columns are pivots, the vectors are linearly independent.
\end{itemize}

\xrule{red} Given $m$ linear combinations over $k$ linearly independent vectors $\bm{b}_1, \dots, \bm{b}_k \in V$.

\begin{equation}
	\begin{aligned}
		\bx_1 &= \sum\limits_{i=1}^k \lambda_{i1} \bm{b}_i \\
		\vdots & \\
		\bx_m &= \sum\limits_{i=1}^k \lambda_{im} \bm{b}_i \\
	\end{aligned}
\end{equation}
%
We can write, with $\B = [\bm{b}_1, \dots, \bm{b}_k]$, the following:
%
\begin{equation}
	\bx_j = \bm{B} \bm{\lambda}_j, \ \ \bm{\lambda}_j = \begin{bmatrix}
		\lambda_1j \\
		\vdots \\
		\lambda_kj
	\end{bmatrix}, \ \ j = 1, \dots,m 
\end{equation}
%
We can test whether $\bx_1, \dots, \bx_m$ are linearly independent using:
\begin{equation}
	\sum\limits_{j=1}^m \psi_j \bx_j = \sum\limits_{j=1}^m \B\bm{\lambda}_j = \B \sum\limits_{j=1}^m \psi_j \bm{\lambda}_j
\end{equation}
%
Which means that $\{\vecs{x}{k}\}$ is linearly independent if the column vectors $\{\vecs{\lambda}{m}\}$ are linearly independent.

\xrule{red} In a vector space $V$, $m$ linear combinations of $\vecs{b}{k}$ are linearly independent if $m > k$. 

%Basis and rank%
\section{Basis and rank}

In a vector space $V$, we are interested in a set of vectors $\mathcal{A}$ that posess the property that any vector $\bm{v} \in V$ can be obtained through a linear combination of vectors in $\mathcal{A}$.

%Generating Set and Basis%
\subsection{Generating Set and Basis}

\xdefinition{red}{Generating Set and Span} Given a set of vectors $\mc{A} = \{ \vecs{x}{k} \} \subseteq \mc{V}$. If every vector $\bm{v} \in \mc{V}$ can be expressed as a linear combination of vectors in $\mc{A}$, $\mc{A}$ is called a \textit{generating set} of $V$.	The set of all linear combinations of vectors in $\mc{A}$ is called the span of $\mc{A}$.

\xdefinition{red}{Basis} Consider a vector space $V = (\mc{V}, +, \cdot)$ and $\mc{A} \subseteq \mc{V}$. A generating set $\mc{A}$ of $\mc{V}$ is called \textit{minimal} if there exists no smaller set $\tilde{\mc{A}} \subsetneq \mc{A} \subseteq \mc{V}$ that spans $\mc{V}$. Every linearly independent generating set of V is minimal and is called a \textit{basis} of $V$.

\xrule{red} Given $\mc{B} \subseteq \mc{V}, \mc{B} \neq \emptyset$. Then the following statements are equivalent:
%
\begin{itemize}
	\item $\mc{B}$ is a basis of $V$
	\item $\mc{B}$ is a minimal generating set.
	\item $\mc{B}$ is a maximal linearly independent set of vectors in $V$ such that adding any vector to the set $\mc{B}$ would make it linearly dependent. 
	\item Every vector $\bm{x} \in V$ is a linear combination of vectors from $\mc{B}$, and every combination is unique, i.e., with
		\begin{equation}
			\bm{x} = \sum\limits_{i=1}^{k} \lambda_i \bm{b}_i
		\end{equation}
\end{itemize}
%

\end{document}
























































